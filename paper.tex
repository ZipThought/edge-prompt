%% BASED ON SAMPLE FILE, TODO: CLEAN UP
%%
%% This is file `sample-sigconf.tex',
%% generated with the docstrip utility.
%%
%% The original source files were:
%%
%% samples.dtx  (with options: `all,proceedings,bibtex,sigconf')
%% 
%% IMPORTANT NOTICE:
%% 
%% For the copyright see the source file.
%% 
%% Any modified versions of this file must be renamed
%% with new filenames distinct from sample-sigconf.tex.
%% 
%% For distribution of the original source see the terms
%% for copying and modification in the file samples.dtx.
%% 
%% This generated file may be distributed as long as the
%% original source files, as listed above, are part of the
%% same distribution. (The sources need not necessarily be
%% in the same archive or directory.)
%%
%%
%% Commands for TeXCount
%TC:macro \cite [option:text,text]
%TC:macro \citep [option:text,text]
%TC:macro \citet [option:text,text]
%TC:envir table 0 1
%TC:envir table* 0 1
%TC:envir tabular [ignore] word
%TC:envir displaymath 0 word
%TC:envir math 0 word
%TC:envir comment 0 0
%%
%% The first command in your LaTeX source must be the \documentclass
%% command.
%%
%% For submission and review of your manuscript please change the
%% command to \documentclass[manuscript, screen, review]{acmart}.
%%
%% When submitting camera ready or to TAPS, please change the command
%% to \documentclass[sigconf]{acmart} or whichever template is required
%% for your publication.
%%
%%
\documentclass[sigconf]{acmart}

%% Inline enumeration
\usepackage[inline]{enumitem}

%%
%% \BibTeX command to typeset BibTeX logo in the docs
\AtBeginDocument{%
  \providecommand\BibTeX{{%
    Bib\TeX}}}

%% Rights management information.  This information is sent to you
%% when you complete the rights form.  These commands have SAMPLE
%% values in them; it is your responsibility as an author to replace
%% the commands and values with those provided to you when you
%% complete the rights form.
\setcopyright{acmlicensed}
\copyrightyear{2025}
\acmYear{2025}
\acmDOI{XXXXXXX.XXXXXXX}
%% These commands are for a PROCEEDINGS abstract or paper.
\acmConference[PromptEng'25]{2nd PromptEng Workshop at the ACM WebConf'25}{April 28--29,
  2025}{Sydney, NSW}
%%
%%  Uncomment \acmBooktitle if the title of the proceedings is different
%%  from ``Proceedings of ...''!
%%
%%\acmBooktitle{Woodstock '18: ACM Symposium on Neural Gaze Detection,
%%  June 03--05, 2018, Woodstock, NY}
\acmISBN{978-1-4503-XXXX-X/2025/04}


%%
%% Submission ID.
%% Use this when submitting an article to a sponsored event. You'll
%% receive a unique submission ID from the organizers
%% of the event, and this ID should be used as the parameter to this command.
%%\acmSubmissionID{123-A56-BU3}

%%
%% For managing citations, it is recommended to use bibliography
%% files in BibTeX format.
%%
%% You can then either use BibTeX with the ACM-Reference-Format style,
%% or BibLaTeX with the acmnumeric or acmauthoryear sytles, that include
%% support for advanced citation of software artefact from the
%% biblatex-software package, also separately available on CTAN.
%%
%% Look at the sample-*-biblatex.tex files for templates showcasing
%% the biblatex styles.
%%

%%
%% The majority of ACM publications use numbered citations and
%% references.  The command \citestyle{authoryear} switches to the
%% "author year" style.
%%
%% If you are preparing content for an event
%% sponsored by ACM SIGGRAPH, you must use the "author year" style of
%% citations and references.
%% Uncommenting
%% the next command will enable that style.
%%\citestyle{acmauthoryear}


%%
%% end of the preamble, start of the body of the document source.
\begin{document}

%%
%% The "title" command has an optional parameter,
%% allowing the author to define a "short title" to be used in page headers.
\title{EdgePrompt: Engineering Guardrail Techniques for Offline LLMs in K-12 Educational Settings}

%%
%% The "author" command and its associated commands are used to define
%% the authors and their affiliations.
%% Of note is the shared affiliation of the first two authors, and the
%% "authornote" and "authornotemark" commands
%% used to denote shared contribution to the research.
\author{Riza Alaudin Syah}
\authornote{Both authors contributed equally to this research.}
\email{alaudinsyah@graduate.utm.my}
\affiliation{%
  \institution{Universiti Teknologi Malaysia}
  \city{Johor Bahru}
  \country{Malaysia}
}

\author{Christoforus Yoga Haryanto}
\authornotemark[1]
\email{cyharyanto@zipthought.com.au}
\affiliation{%
  \institution{ZipThought}
  \city{Melbourne}
  \state{VIC}
  \country{Australia}
}

\author{Emily Lomempow}
\affiliation{%
 \institution{ZipThought}
 \city{Melbourne}
 \state{VIC}
 \country{Australia}
}

\author{Krishna Malik}
\affiliation{%
  \institution{Independent Researcher}
  \city{Jakarta}
  \country{Indonesia}
}

\author{Irvan Putra}
\affiliation{%
  \institution{Independent Researcher}
  \city{Jakarta}
  \country{Indonesia}
}

%%
%% By default, the full list of authors will be used in the page
%% headers. Often, this list is too long, and will overlap
%% other information printed in the page headers. This command allows
%% the author to define a more concise list
%% of authors' names for this purpose.
\renewcommand{\shortauthors}{Syah et al.}

%%
%% The abstract is a short summary of the work to be presented in the
%% article.
\begin{abstract}
EdgePrompt is a prompt engineering framework that implements pragmatic guardrails for Large Language Models (LLMs) in K-12 educational settings through structured prompting inspired by neural-symbolic principles. The system addresses educational disparities in Indonesia's Underdeveloped, Frontier, and Outermost (3T) regions by enabling offline-capable content safety controls. It combines: (1) content generation with structured constraint templates, (2) assessment processing with layered validation, and (3) lightweight storage for content and result management. The framework extends existing initiatives by implementing a multi-stage verification workflow that maintains safety boundaries while preserving model capabilities in connectivity-constrained environments. Initial deployment targets Grade 5 language instruction, demonstrating effective guardrails through structured prompt engineering without requiring formal symbolic reasoning components.
\end{abstract}

%%
%% The code below is generated by the tool at http://dl.acm.org/ccs.cfm.
%% Please copy and paste the code instead of the example below.
%%
\begin{CCSXML}
<ccs2012>
   <concept>
       <concept_id>10003456.10003457.10003527.10003541</concept_id>
       <concept_desc>Social and professional topics~K-12 education</concept_desc>
       <concept_significance>500</concept_significance>
    </concept>
    <concept>
       <concept_id>10010405.10010489.10010490</concept_id>
       <concept_desc>Applied computing~Computer-assisted instruction</concept_desc>
       <concept_significance>500</concept_significance>
    </concept>
    <concept>
       <concept_id>10010147.10010178.10010179.10010182</concept_id>
       <concept_desc>Computing methodologies~Natural language generation</concept_desc>
       <concept_significance>300</concept_significance>
    </concept>
 </ccs2012>
\end{CCSXML}

\ccsdesc[500]{Social and professional topics~K-12 education}
\ccsdesc[500]{Applied computing~Computer-assisted instruction}
\ccsdesc[300]{Computing methodologies~Natural language generation}

%%
%% Keywords. The author(s) should pick words that accurately describe
%% the work being presented. Separate the keywords with commas.
\keywords{Large Language Models, Edge Computing, K-12 Education, AI Safety, Prompt Engineering, Content Filtering, Offline AI, Educational Technology, Guardrails}
%% A "teaser" image appears between the author and affiliation
%% information and the body of the document, and typically spans the
%% page.
%%\begin{teaserfigure}
%%  \includegraphics[width=\textwidth]{sampleteaser}
%%  \caption{Seattle Mariners at Spring Training, 2010.}
%%  \Description{Enjoying the baseball game from the third-base
%%  seats. Ichiro Suzuki preparing to bat.}
%%  \label{fig:teaser}
%%\end{teaserfigure}

%% \received{13 January 2025}
%% \received[accepted]{XX January 2025}
%% \received[revised]{XX February 2025}

%%
%% This command processes the author and affiliation and title
%% information and builds the first part of the formatted document.
\maketitle

\section{Introduction}

Recent advances in guardrail implementations for Large Language Models (LLMs) have demonstrated promising domain-specific control mechanisms  \cite{dong_building_2024, dong_safeguarding_2024}. In resource-constrained educational environments, however, implementing effective guardrails requires balancing safety constraints with offline operational capabilities. We define guardrails as "structured prompt-based controls enforcing content boundaries while maintaining validation across edge deployments" through:
\begin{enumerate*}
    \item \textbf{Structured Prompting:} templates encoding safety constraints,
    \item \textbf{Multi-stage Validation:} sequential prompt-based checks, and
    \item \textbf{Edge Deployment Compatibility:} mechanisms for resource-constrained operation
\end{enumerate*}.

Drawing from neural-symbolic architectures \cite{dong_safeguarding_2024} while operating within prompt engineering bounds, EdgePrompt implements pragmatic guardrails for K-12 assessment systems that require freeform answers, beyond multiple choice questions \cite{hang_mcqgen_2024}. Our framework addresses core technical challenges:
\begin{enumerate*}
    \item maintaining content safety without persistent connectivity,
    \item enabling sophisticated assessment in edge deployments, and
    \item ensuring consistent validation across distributed components
\end{enumerate*}.

The implementation demonstrates offline-capable content filtering through:
\begin{enumerate*}
    \item cloud/edge generation with material-scoped guardrails,
    \item edge evaluation supporting offline operation, and
    \item atomic storage operations
\end{enumerate*}.
Initial deployment targets Grade 5 language instruction, emphasizing practical validation in resource-limited settings.

Indonesia has more than 17,000 islands and most of its territory is seas, thus internet penetration is quite challenging, especially for accessing cloud-based LLM tools like ChatGPT. The Indonesian “3T” areas—Frontier, Outermost, and Disadvantaged—face economic, infrastructure, and human resource challenges, requiring targeted government development programs \cite{kementerian_desa_official_2025, noauthor_peraturan_2020}. Edge computing and LLMs can transform education in 3T areas by enabling offline AI-powered learning with minimal internet dependency. LLMs can provide localized, adaptive tutoring, while edge devices enhance access to digital resources, bridging educational gaps and fostering inclusive, remote learning opportunities in underserved regions.

\section{Methodology}

Our system implements a rigidly structured validation pipeline with strategically placed guardrails that leverage cloud and edge LLMs for distinct operational roles. The architecture enforces safety through multi-stage template validation, explicit constraint propagation, and formalized evaluation protocols, as shown in Fig. \ref{fig:teaser}. Due to the offline focus of the evaluation infrastructure, we are choosing edge-deployment-capable LLM. We're focusing on the prompt engineering strategy and comparison across different guardrails prompting techniques. The differences between LLM selections are out-of-scope.

\subsection{Teacher-Driven Content Generation}

\begin{enumerate}
    \item \textbf{Question Template Definition:}
        \begin{enumerate*}
            \item domain-constrained content templates $T_c$,
            \item answer space specification $A_s$ with explicit boundaries, and
            \item formal learning objective mapping $O: T_c \rightarrow L$ where $L$ defines permissible learning outcomes.
        \end{enumerate*}
    \item \textbf{Cloud/Edge LLM Assistance Pipeline:}
        \begin{enumerate*}
            \item rubric formalization function $R(c_t, v_p)$ where $c_t$ represents teacher criteria and $v_p$ validation parameters,
            \item transformation $S: R \rightarrow R'$ ensuring edge compatibility, and
            \item grading template generation $G(R')$ with explicit validation constraints.
        \end{enumerate*}
\end{enumerate}

\subsection{Student Answers Evaluation Infrastructure}

\begin{enumerate}
    \item \textbf{Edge Validation Protocol:}
        \begin{enumerate*}
            \item verification $V(q, a) \rightarrow \{0,1\}$ for question-answer pairs alignment,
            \item staged response validation sequence $\{v_1, ..., v_n\}$ against rubric $R'$, and
            \item boundary enforcement function $B(r) \rightarrow \{valid, invalid\}$ for responses $r$.
        \end{enumerate*}
    \item \textbf{Evaluation Logic:}
        \begin{enumerate*}
            \item application of $R'$ through transformation $E(r, R')$,
            \item calibrated scoring function $S(e)$ for evaluation $e$, and
            \item constraint satisfaction verification $C(s, c_t)$ for score $s$.
        \end{enumerate*}
\end{enumerate}

\subsection{Teacher Verification Protocol}

\begin{enumerate}
    \item \textbf{Response Analysis:}
        \begin{enumerate*}
            \item edge case detection function $D(r, \theta)$ with threshold $\theta$,
            \item review trigger $T(d) \rightarrow \{review, accept\}$, and
            \item calibration state tracking $K(h)$ over evaluation history $h$.
        \end{enumerate*}
    \item \textbf{System Adaptation:}
        \begin{enumerate*}
            \item rubric adjustment $A: R' \rightarrow R''$,
            \item criteria optimization function $O(K, \epsilon)$ with convergence parameter $\epsilon$, and
            \item template refinement process $P(T_c, h)$ based on performance history.
        \end{enumerate*}
\end{enumerate}

\section{Implementation Strategy}

Our technical implementation combines rigorous validation protocols with pragmatic deployment considerations:

\subsection{Core Components}

\begin{enumerate*}
    \item \textbf{Template Processing:}
        \begin{enumerate*}
            \item prompt template definition $T(p, c)$ encoding patterns $p$ and constraints $c$,
            \item validation rule formalization $V(r)$ for rubric $r$, and
            \item edge-compatible transformation protocols
        \end{enumerate*}
    \item \textbf{Validation Framework:}
        \begin{enumerate*}
            \item constraint checking $C(i, r)$ for input $i$,
            \item staged response validation $\{v_1,...,v_n\}$, and
            \item boundary enforcement $B(r) \rightarrow \{valid, invalid\}$
        \end{enumerate*}
    \item \textbf{Integration Architecture:}
        \begin{enumerate*}
            \item state synchronization,
            \item atomic storage, and
            \item failure recovery
        \end{enumerate*}
\end{enumerate*}

Example prompts can be seen in the appendix.

\subsection{Development Strategy}
The deployment strategy would follow several prior similar activities that rely on physical package distribution by using commercial logistics vendors. The general deployment plan for EdgeLLM:
\begin{enumerate*}
    \item build compact all-in-on mini PC that is capable of running on device LLM like LLama 3.2 8B,
    \item setup all necessary hardware and software,
    \item package it properly and send it via a logistics vendor, and
    \item gather online unboxing and hands-on tutorial with the recipient.
\end{enumerate*}


\subsection{Validation Strategy}

Technical validation emphasizes three key aspects:
\begin{enumerate}
    \item \textbf{Functional Metrics:}
        \begin{enumerate}
            \item guardrail effectiveness $E(g) = \frac{\text{valid\_responses}}{\text{total\_responses}}$,
            \item offline stability $S(t) = \frac{\text{successful\_operations}}{\text{total\_operations}}$, and
            \item teacher workflow integration rate
        \end{enumerate}
    \item \textbf{Performance Analysis:}
        \begin{enumerate}
            \item edge resource utilization $U(r) = \frac{\text{used\_resources}}{\text{available\_resources}}$,
            \item response latency distribution $L(t)$, and
            \item throughput scaling characteristics
        \end{enumerate}
    \item \textbf{System Insights:}
        deployment optimization, constraint analysis, and scaling considerations
\end{enumerate}

Our evaluation framework integrates established guardrail effectiveness metrics \cite{niknazar_building_2024} with edge-specific performance indicators, providing quantitative validation of the system's practical viability. These results will inform subsequent publications exploring comprehensive system evaluation and broader educational applications.

\section{Resources}
In the appendix, sample prompts can be seen in Fig. Fig. \ref{fig:prompt_bahasa} and \ref{fig:prompt_english}.
Repository: \url{https://github.com/build-club-ai-indonesia/edge-prompt}
Data source: \url{https://buku.kemdikbud.go.id/}
  

%%
%% The acknowledgments section is defined using the "acks" environment
%% (and NOT an unnumbered section). This ensures the proper
%% identification of the section in the article metadata, and the
%% consistent spelling of the heading.
\begin{acks}
\href{https://www.buildclub.ai/}{BuildClub.ai} as the training campus for AI learners, experts, builders.
\end{acks}

%%
%% The next two lines define the bibliography style to be used, and
%% the bibliography file.
\bibliographystyle{ACM-Reference-Format}
\bibliography{references}


%%
%% If your work has an appendix, this is the place to put it.
\appendix

\begin{figure*}[t]
    \centering
    \includegraphics[width=\textwidth]{seqdiagram.png}
    \caption{Sequence diagram of the system.}
    \Description{Sequence diagram of the system.}
    \label{fig:teaser}
\end{figure*}

\begin{figure}
    \centering
    \includegraphics[width=\columnwidth]{image.png}
    \caption{Sample prompt in Bahasa Indonesia}
    \label{fig:prompt_bahasa}
\end{figure}

\begin{figure}
    \centering
    \includegraphics[width=\columnwidth]{image2.png}
    \caption{Sample prompt in English}
    \label{fig:prompt_english}
\end{figure}

\end{document}
\endinput
%%
%% End of file `sample-sigconf.tex'.
